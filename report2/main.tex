\documentclass[a4paper]{article}

%% Language and font encodings
\usepackage[english]{babel}
\usepackage[utf8x]{inputenc}
\usepackage[T1]{fontenc}
%% Sets page size and margins
\usepackage[a4paper,top=3cm,bottom=2cm,left=3cm,right=3cm,marginparwidth=1.75cm]{geometry}

%% Useful packages
\usepackage{amsmath}
\usepackage{graphicx}
\usepackage[colorinlistoftodos]{todonotes}
\usepackage[colorlinks=true, allcolors=blue]{hyperref}
\usepackage{booktabs}
\title{Heuristic Optimization : Implementation exercise 2}
\date{May 17, 2017}
\author{Nikita Marchant (nimarcha@vub.ac.be)}

\setlength{\parskip}{7pt}%
\setlength{\parindent}{0pt}%
\renewcommand{\baselinestretch}{1}

\begin{document}
\maketitle

\section{Introduction}

This document is the report the implementation exercise 2 for the Heuristic Optimization course.
The exercise asks to implement two stochastic local search algorithms for the the permutation flow-shop problem with the sum weighted completion time objective (also called PFSP-WCT).

The PFSP-WCT has not been thoroughly studied in the literature so i used some inspiration from papers studying the PFSP with flowtime that is more studied.

\section{Algorithms}

The first algorithm is an Iterated Local Search (ILS) inspired from \cite{panruiz2012} and the second is a genetic algorithm inspired from \cite{zhang2009}.

\subsection{Iterated Local Search}

The Iterated Local Search is a stochastic meta-heuristic
known to be applicable to multiple optimization problems~\cite{Lourenco2010}.

It consists of these steps :

\begin{enumerate}
    \item Generate a initial solution
    \item Apply a local search to the solution
    \item When stuck in a local optimum, apply a perturbation
    \item Apply a local search to the solution
    \item Use an acceptance criterion for the perturbed then optimized solution
    \item Go back to 3 or stop when a termination criterion is met.
\end{enumerate}

\subsubsection{Initialization method: $LR(x)$}
\label{LR}

The initialization method used in this implementation and in \cite{panruiz2012} is the $LR(x)$ heuristic introduced by \cite{liu2001}.  It consist in three steps :
\begin{enumerate}
    \item Rank the jobs by their weighed sum of flowtime
    \item Generate $x$ solutions by inserting the job from position $x$ at the front of the solution
    \item Select the sequence with the minimum weighted flow time
\end{enumerate}

\subsubsection{Local search}
\label{RZ}
The local search that was implemented is the iterated RZ (IRZ). The RZ uses a insertion neighborhood.
It sequentially inserts each job at each possible position in the candidate solution (thus is in $O(n^2)$ complexity if $n$ is the number of jobs) and keeps the best one.

The IRZ applies RZ until a local optimal solution is found (i.e. the RZ does not yield a better result).

\subsubsection{Perturbation method}
\label{pertu}

After finding a local optimum, a perturbation is applied to be able to escape the local optimum and extend the search space.

The perturbation method consists of $\gamma$ random insertion moves : each move selects randomly a job and moves it to a random position.

\subsubsection{Acceptance criterion}

After finding a local optimum, we have to decide if we keep the solution. I chose to implement the {simulated annealing} as the criterion with $\lambda \cdot \frac{\sum_{j=1}^{n} \sum_{i=1}^{m} p_{ij}}{10 mn}$ as a constant temperature.

\subsubsection{Termination criterion}

For both algorithm, the termination criterion is the time. For instances of size $n=50$, i stop at 70 seconds and for $n=100$ i stop at 200 seconds. Theses values were chosen as 100 times the runtime of the algorithms implemented in the first part.


\subsection{Genetic algorithm}

My implementation of a genetic algorithm is inspired form \cite{zhang2009} with some slight differences.

\begin{enumerate}
    \item Generate a initial population
    \item Generate a new generation with the crossover operator
    \item Perturbate each chromosome with a given probability
    \item Apply a local search on each chromosome
    \item Merge the old and the new population and select the eligible chromosomes to be kept
    \item Goto 2 or stop if the termination criterion is met
\end{enumerate}

\subsubsection{Generation of the initial population}

The initial population consists of one chromosome generated with $LR(x)$ (see \ref{LR}) and the other
are generated as random permutation of the job set.

\subsubsection{Crossover operator}


\subsubsection{Mutation operator}

The mutation operator if the same as in \ref{pertu} and if applied only with $P_m \cdot \sqrt{U + 1}$ probability.
$U$ is defined as the number of generation since the last improvement to the global solution. This is not present in \cite{zhang2009} but helps to unstuck the algorithm when it is stuck in a local optimum since a long time.


\subsubsection{Local search}

The local search method is RZ (as explained in \ref{RZ}) but also as an original addition, if $U$ is bigger than $U_{IRZ}$, a IRZ algorithm is used instead because the mutation rate is much higher so the chromosome is highly perturbed.

Using a IRZ all the time was tried but yielded no better results and was significantly slower\footnote{Due to the lack of computing power (see \ref{power}), this hypothesis could not be formally proved.}.

\subsubsection{Population selection}



\section{Parameters}
\subsection{Computation power limitation}
\label{power}
\subsection{Iterated Local Search}
\subsection{Genetic algorithm}

\section{Results}


\section{Implementation}

\section{Conclusion}

\bibliographystyle{apalike}
\bibliography{bib}

\end{document}
